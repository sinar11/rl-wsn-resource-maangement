In wireless sensor networks, resource-constrained
nodes are expected to operate in highly dynamic and
often unattended environments. Hence, support for
intelligent, autonomous, adaptive and distributed
sensor management is an essential ingredient of a
middleware solution for development of scalable and
dynamic wireless sensor network applications. We
present reinforcement learning based sensor
management framework to enable autonomous
self-learning/adaptive applications with inherent
support for efficient resource management. Our
design goal is to create a system using a bottom-up
approach where each sensor node is responsible
for its resource allocation/task selection
while optimizing global system-wide
parameters like total energy usage, network lifetime
etc.\\
We first present Distributed Independent
Reinforcement Learning (DIRL), a Q-learning based
scheme that learns the utility of performing
various tasks over time using mostly local
information at nodes and uses the utility value
along with application constraints for task
management by optimizing parameters like total
energy usage. We next extend above scheme to create
a two-tier reinforcement learning based framework
where first tier learning (micro-learning) enables
individual sensor nodes to self-schedule their
tasks using local information allowing for a
real-time adaptation as in DIRL. Second tier
learning (macro-learning) governs the
micro-learners by setting their utility functions
in order to steer the system towards application�s
global optimization goal (e.g. maximize network
lifetime etc). We exemplify the effectiveness of
our framework by designing a tracking/surveillance
application on top of it. Finally, we present
results of simulation studies to compare
performance of our scheme against other existing
approaches. In general for applications requiring
autonomous adaptation, we show that our two-tier
reinforcement learning based scheme on average is
about 50\% more efficient than micro-learning alone
and many fold more efficient than traditional
resource management schemes like static scheduling,
while maintaining necessary accuracy/performance. \\
We also address the issue of efficient data
collection in sparse wireless sensor networks.
Sparse WSNs are being effectively used in several
monitoring applications in farms, battle-fields,
urban traffic and the environment. We investigate
data collection, a crucial task in sparse WSNs by
means of special nodes called Mobile Data
Collectors (MDCs), which visit sensor nodes to
gather data. As contact times are not known a
priori, the discovery of an incoming MDC by the
static sensor node is a critical task. Ideally, a
discovery strategy should be able to correctly
predict contacts in order to minimize energy
consumption. Discovery becomes more challenging as
MDCs participating in various applications exhibit
different mobility patterns and hence requires
unique design of discovery strategy for each
application. In this paper, we propose an adaptive
discovery strategy that exploits state-based model
free reinforcement learning to create a generic
framework that can be effectively applied to
various applications while minimizing energy
consumption. The principal idea is to learn
underlying pattern of MDCs' arrival and tune sensor
node's duty cycle accordingly. We carry out an
extensive simulation analysis to demonstrate the
energy efficiency and effectiveness of the proposed
framework. Simulation results show that our
solution provides superior performance in terms of
both discovery efficiency and energy conservation
and can be directly applied to demonstrated
applications of sparse WSNs.\\ 
Finally, we investigate development of a complete
and generalized sensor management framework
by using directed diffusion and our multi-layer
reinforcement learning scheme. The goals of this
work are two-fold: to incorporate effective and proven task (interest) and data dissemination techniques
of directed diffusion in a resource management
framework; and to improve efficiency and autonomous
adaptation of directed diffusion in sensor
selection and scheduling. Our studies demonstrate
how individual node level as well as neighborhood
and global level learning can help optimize the
system in a generic manner while maintaining
robust, localized and distributed sensing as
provided by directed diffusion. Using performance
analysis it is shown that the proposed approach for
adaptive management and task scheduling results in
up to 100\% increase in system lifetime compared to
traditional directed diffusion.


