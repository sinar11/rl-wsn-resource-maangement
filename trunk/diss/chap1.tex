\chapter{INTRODUCTION}

% \pagestyle{plain}

% no \PARstart
In recent years, wireless sensor networks (WSN)
have become increasingly popular in playing the
role of real world modalities. A WSN consists of a
large number of tiny sensor nodes capable of
interfacing with the real world, communicating
wirelessly and performing limited computation.
These tiny sensor nodes are low cost, low power and
are easily deployable and can be readily networked
and hence offers opportunity for a wide range of
applications. Most popular areas for WSN
applications are health, military, environment,
home and security.
\section{Scope and Challenges}
\label{scope-challenges}
Due to tight integration with the physical world,
sensor networks present significant challenges to
the application development. As a result, most of
the WSN applications till date focus on simple
data-gathering and control designed to work on
specific network and hence resulting in monolithic
applications strongly coupled with protocols and
network used. But as sensor network applications
become pervasive, such monolithic and ad hoc
approaches does not work and a systematic
application design method based on standards and
higher level abstractions is required [22]. To
simplify sensor network application development,
the need for programming abstractions such as a
middleware framework is well acknowledged. WSN
middleware can be considered as a software
infrastructure that resides above operating system
and below the application abstracting low-level
functionality by providing high-level abstractions.
A complete middleware solution should provide a
holistic view of the network, services for data
collection, aggregation and control while
efficiently and adaptively managing system
resources as per application's requirements. \\
% % % Resource management (DIRL+COIN)
WSN nodes are remarkably constrained in terms of
their resources viz. energy, computational power
and radio bandwidth. WSNs normally operate in
uncertain and dynamic environments where the state
of the system changes considerably over time. For
example, in data collection applications,
uncertainty exists due to intermittent links or
traffic conditions. Moreover, the network itself is
dynamic due to such events as node mobility and
depleted battery. WSN applications need to cope
with such dynamics and uncertainty inherent in
sensor networks, while simultaneously trying to
achieve application's requirements for QoS and
optimization goal. Consequently, adaptive resource
management is a key to any successful middleware
solution enabling such applications.\\
Resource management includes initial
sensor-selection and task allocation as well as
runtime adaptation of allocated task/resources.
There are many proposed middleware solutions that
have advocated strong need for proactive adaptation
of resources [5,14,28], but there are only few that
have actually tried to resolve the issue of
enabling adaptive resource management for WSN
applications. This problem of resource
management/adaptation can be described as follows:
\\ \emph{Given application structure, QoS
requirements and current system state, what is the
best way of allocating tasks to resources so that
given system-wide, application-driven, global
parameters can be optimized.}\\ In the above, the
application structure is in the form of underlying
tasks and their interactions. QoS requirements
include such constraints as latency, reliability,
coverage etc. while the current state of the system
is defined by parameters like mobility, energy
availability, and neighboring nodes. The parameters
to be optimized include energy, bandwidth, and
network lifetime. Without loss of generality, we
illustrate the resource management problem using a
simplified object/entity tracking application.
Object tracking application can be considered to
consist of following tasks: 1) sample- sense the
environment (e.g. signal strength of a moving
object). 2) transmit (Tx)- transmit a message to
next hop towards the base-stations. 3) receive
(Rx)- turn radio to receive mode to listen for
incoming messages. 4) aggregate- aggregate two or
more local and remote same target readings into
single reading (e.g. data triangulation for better
position estimation or mapping to a known track or
simply \emph{last value} aggregation function). 5)
sleep- put CPU and radio in sleep mode to minimize
battery consumption. State representation may
consist of the following variables: have one or
more neighbors, successful in recent sampling,
successful in recent receive, signal strength (or
quality of reading). QoS requirements here may
include quality of signal, tracking coverage area
as well as maximum allowed latency. Our goal in
this case is to optimize energy usage among all
sensor nodes. Thus the goal of our resource
management framework is to schedule and allocate
tasks on each sensor node in the system, so that
energy usage among all sensor nodes is minimized
while fulfilling the coverage and latency
requirements of the application.\\
% % % Sparse WSN
Most sensor network applications studied today
consists of a large number of sensor nodes deployed
over a geographical area. Sensors use multi-hop
communication to send data acquired from the
external environment to a sink node or to an Access
Point (AP) in the infrastructure. However, several
applications do not require fine-grained sensing.
Examples of such applications include monitoring of
weather conditions in large areas, air quality in
urban scenarios, terrain conditions for
agriculture, and so on. In this case, it is
possible to consider a \emph{sparse wireless sensor
network}, i.e., a WSN where the density of nodes is
so low that they cannot communicate with each other
through multi-hop paths, or even directly. In order
to make communication feasible, data collection in
sparse WSNs can be accomplished by means of
\emph{mobile data collectors} (MDCs). MDCs are
special mobile nodes responsible for data gathering
and/or dissemination. They are assumed to be
powerful in terms of data storage and processing
capabilities, and are not energy constrained.
However, the data collection paradigm in sparse
WSNs with MDCs is different, and introduces
significant challenges including contact detection
and energy conservation. \\
Communication between an MDC and each sensor node
takes place in two phases. First, sensor node
discovers the presence of the MDC in its
communication range. Then, it transfers collected
data to the MDC while satisfying any required
reliability constraints. Unlike MDCs, sensor nodes
have a limited energy budget, so that the
data-collection process has to be energy efficient
in order to prolong their network lifetime. In
addition, such energy-conserving mechanisms should
not compromise the timeliness of communication.
This is critical especially when the MDC has only a
short contact time with sensors, and also in the
case when such contacts cannot be predicted
accurately. In fact, a major problem in data
collection is that sensor nodes usually do not have
\emph{a priori} knowledge of the MDC mobility
pattern. Furthermore, even in cases where the
arrivals can be predicted, there is a chance that
the MDC contacts can be affected by delays or can
change their rate. Hence robust and flexible
mechanisms have to be defined in order to adapt to
operating conditions autonomously.\\
% % Communication Paradigm
Support for a scalable and robust communication
paradigm is another important characteristic of a
middleware solution. As all sensor network
applications are data-driven, middleware should
ideally support a data-centric model for task and
data dissemination. Middleware should allow
application specific in-network processing and data
filtering to enable application assisted routing,
aggregation, data fusion and collaborative
information processing.
\section{Contributions}
\label{contributions}
We next introduce our contributions in this
dissertation addressing the issues of developing a
dynamic, adaptive and autonomous middleware for WSN
management. %
\subsection{Adaptive Resource Management Using Distributed Independent Reinforcement Learning}\label{intro:dirl}
Our first contribution is a Distributed Independent
Reinforcement Learning (DIRL) based scheme for WSN
resource management. The main idea of DIRL is to
allow each individual sensor node to self-schedule
its tasks and allocate its resources by learning
their usefulness (utility) in any given state while
honouring application defined constraints and
maximizing total amount of reward over time. DIRL
is based on Q-learning [17], a form of model-free
reinforcement learning. Q-learning is quite simple,
demands minimal computational resources and doesn�t
require a model of the environment in order to
operate. Hence it is ideal for implementation on
resource-constrained sensor nodes. DIRL uses
classic exploration and exploitation strategy as
used in most RL based approaches to learn utilities
of various tasks. DIRL addresses structural credit
assignment (propagation of reward spatially across
states in order to define notion of similar states)
problem by using weighted hamming distance between
two states. Here, the application state is
represented in the form of system and application
variables each carrying an associated weight. DIRL
employs independent learning where each agent
applies the learning algorithm in a classic sense
(like single agent system) ignoring the presence of
other agents. The main advantage of using
independent learning in DIRL is that no
communication is required for co-ordination between
sensor nodes and each node selfishly tries to
maximize its own rewards. %
\subsection{Ensuring Global Optimization With Multi-Tier Reinforcement Learning}\label{intro:coin}
DIRL works well when each node in WSN application
is acting of its own and doesn�t need to co-operate
or compete with other nodes. In other words, if all
nodes are acting independently and their actions do
not affect others, then any increase in a node�s
utility cannot decrease anyone else�s utility and
hence will always increase world (system-wide)
utility which is merely sum of all node�s utilities
over all times. Such a system is sub-world factored
and will eventually attain a Pareto-optimal point
and hence towards our system-wide optimization
goal. But most of the real-world WSN applications
need some sort of co-operation among sensor nodes
and hence nodes cannot work independently. In this
case, increase in utility of individual node may
result in reduction of other node�s utility and
hence may not increase world utility. It is also
possible that such system can lead to phenomena
like Tragedy of the Commons (TOC) or Braes�
Paradox, wherein individual�s selfishness leads to
significantly lower world utility. Such phenomena
can be avoided by carefully designing agent�s
utility functions as well as constraints under
which agent performs task selection. In other
words, we need to make sure that individual�s
utility is �aligned� with the world utility, i.e.
any increase in agent�s private utility because of
its action will also result in increase of world
utility. COllective INtelligence (COIN) theory
provides principles on designing such private (individual node) as well as
global utility functions such that
they are aligned. We follow these principles to
design a multi-tier reinforcement learning based
framework for WSN resource management.\\
Our design goal is to create a system using a
bottom-up approach where each sensor node is
responsible for task selection, rather than
top-down approach (where some central entity
dictates nodes what task to execute) used by many
other middleware solutions [Heinzelman04, Yu04].
Main advantages of bottom-up approach are
pro-active and real-time adaptation, no centralized
processing requirement for task allocation and
minimal communication overhead. But principal
challenge of bottom-up approach is how to make sure
that system is actually meeting the global
application goals and is not just acting randomly
or creating chaos. We resolve this issue by using
two-layer learning: micro-learning as used by
individual nodes to self-schedule their tasks and
macro-learning as used by each data-stream subworld
to steer the system towards application goal by
setting/updating rewards for micro-learners.
%
\subsection{Resource-aware Data Collection in Sparse Wireless Sensor Networks}\label{intro:dirl-sparse}
Next we explore the issue of energy-aware resource
allocation in sparse WSNs with Mobile Data
Collectors (MDCs). We define discovery and data
transfer protocols for energy-efficient data
collection in sparse WSNs with MDCs, and propose an
adaptive strategy exploiting our Distributed
Independent Reinforcement Learning scheme. We
design a generic adaptive data collection (ADC)
framework that can be applied to wide range of
applications while minimizing energy consumption.
The principal idea is to learn underlying pattern
of MDCs' arrival and tune sensor node's duty cycle
accordingly. Through extensive simulations, we show
that our framework is highly efficient in terms of
low duty cycle, high discovery rate and high energy
and data transfer efficiency.
%
\subsection{Design of Adaptive Middleware Using Directed Diffusion}\label{intro:diffusion}
%
\section{Organization of Dissertation}\label{organization}
In chapter 2, we review existing literature and provide some necessary background
on a few related topics that have been used in
this dissertation. Chapter 3 describes our
preliminary DIRL scheme for resource management
in WSN. We present our extended multi-tier
reinforcement learning based framework
using COIN and DIRL in Chapter 4. In chapter 5, we
explore application of DIRL to sparse WSN for
energy-aware MDC discovery. Chapter 6 provides a
detail design of our complete adaptive and
autonomous middleware.
